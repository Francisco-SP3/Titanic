\documentclass{article}


\usepackage{amsmath}
\usepackage{multicol}
\usepackage{hyperref}
\usepackage{float}
\usepackage[spanish]{babel}
\usepackage[utf8]{inputenc}
\usepackage[a4paper,left=3.17cm,right=3.17cm,top=2.54cm,bottom=2.54cm]{geometry}
\usepackage{crop,graphicx,amsmath,array,color,amssymb,fancyhdr,lineno}
\usepackage{flushend,stfloats,amsthm,chngpage,times,,lipsum,lastpage} 
\usepackage{calc,listings,color,wrapfig,tabularx,longtable,enumitem}
\usepackage[style=numeric-comp,backend=biber, sorting=nty]{biblatex}
\usepackage{lineno}
\renewcommand{\baselinestretch}{1.5}
\addbibresource{Refs.bib}

\pagestyle{fancy}
\fancypagestyle{plain}{
  \renewcommand{\headrulewidth}{0pt}
  \fancyhf{}
}

\title{
  \begin{center}
    Reporte: Modelo de Machine Learning para predicción de supervivencia con datos del Titanic \\[0.5 cm]
    
  \end{center}
}

\begin{document}

\author{Francisco Salas Porras}
\begin{titlepage}

    \setlength{\multicolsep}{0pt}% 50% of original values
    \newcommand{\HRule}{\rule{\linewidth}{0.5mm}} % Defines a new command for the horizontal lines, change thickness here

    %----------------------------------------------------------------------------------------
    %	LOGO SECTION
    %----------------------------------------------------------------------------------------
    \centering
    \includegraphics[width=10cm]{Title/logoTec.png}\\[1cm]


    %----------------------------------------------------------------------------------------


    %----------------------------------------------------------------------------------------
    %	HEADING SECTIONS
    %----------------------------------------------------------------------------------------

    \textsc{\LARGE Instituto Tecnológico y de Estudios Superiores de Monterrey }\\[1 cm] % Name of your university/college
    \textsc{\large Escuela de Ingeniería y Ciencias}\\[0.5cm] % Minor heading such as course title
    \textrm{\Large TC3006C.102: Inteligencia artificial avanzada para la ciencia de datos I}\\[0.5cm]
    %----------------------------------------------------------------------------------------
    %	TITLE SECTION
    %----------------------------------------------------------------------------------------
    \makeatletter
    \HRule \\[0.4cm]
    { \LARGE \bfseries \@title \large Report: Machine Learning Model for survival prediction with Titanic data. }\\[0.4cm] % Title of your document
    \HRule \\[0.6 cm]

    %----------------------------------------------------------------------------------------
    %	AUTHOR SECTION
    %----------------------------------------------------------------------------------------


    \begin{multicols}{2}[\Large Autores - Equipo 5]
        \begin{center} \large
            \textrm{
                Francisco Salas Porras - A01177893
                \\
                Jackeline Conant Rubalcava - A01280544
                \\
                J. Andrés Orantes Guillén - A01174130
                \\
                Luis Mario Lozoya Chairez - A00833364
            }
        \end{center}
    \end{multicols}
    \textrm{\large J. Eduardo Corrales Cardoza - A01742328\\[0.6 cm]}

    \begin{multicols}{2}[\Large Profesores]
        \begin{center} \large
            \textrm{
                Alfredo Esquivel Jaramillo\\[0.1em]
                Antonio Carlos Bento\\[0.1em]
                Frumencio Olivas Alvarez\\[0.1em]
                Hugo Terashima Marín\\[0.1em]
                Jesús Adrián Rodríguez Rocha\\[0.1em]
                Julio Antonio Juárez Jiménez
            }
        \end{center}
    \end{multicols}
    \textrm{\large Mauricio González Soto\\[1cm]}

    \makeatother

    % If you don't want a supervisor, uncomment the two lines below and remove the section above
    %\Large \emph{Author:}\\
    %John \textsc{Smith}\\[3cm] % Your name

    %----------------------------------------------------------------------------------------
    %	DATE SECTION
    %----------------------------------------------------------------------------------------
    \vfill

    {\large \ Monterrey, Nuevo León. 31 Junio 2024}\\[0.5 cm] % Date, change the \today to a set date if you want to be precise
    \break
    \vfill % Fill the rest of the page with whitespace

\end{titlepage}

\sffamily

\fancyhf{}
\fancyhead[L]{Reporte Técnico}
\fancyhead[R]{}
\fancyfoot[R]{ \bf\thepage\ \rm }%


\section{Hello siso! money money moneyyy} % creates a section

\section*{Resumen}
El presente documento ahonda en la predicción de la supervivencia de los pasajeros del Titanic utilizando técnicas de aprendizaje automático y una base de datos histórica. A partir de los datos disponibles, se inició con la exploración y limpieza de los datos, abordando valores faltantes y realizando la categorización de variables. Posteriormente, se implementaron técnicas de feature engineering para mejorar la capacidad predictiva del modelo en desarrollo.

Para la selección del modelo, se investigaron y compararon diversos algoritmos de clasificación binaria, como Random Forest, Regresión Logística y XGBoost. Los modelos fueron evaluados utilizando métricas como exactitud, F1-Score y ROC-AUC Score. Los resultados indicaron que el modelo con mejor rendimiento fue el Random Forest, el cual será refinado para optimizar su desempeño.

\section*{Abstract}
This paper delves into the prediction of the survival of Titanic passengers using machine learning techniques and a historical database. Based on the available data, the data was explored and cleaned, addressing missing values and categorizing variables. Subsequently, feature engineering techniques were implemented to improve the predictive capacity of the model under development.

For model selection, various binary classification algorithms were investigated and compared, such as Random Forest, Logistic Regression, and XGBoost. The models were evaluated using metrics such as accuracy, F1-Score, and ROC-AUC Score. The results indicated that the model with the best performance was Random Forest, which will be refined to optimize its performance.


%\input{Main/Introduccion}
%\input{Main/Planteamiento}
%\input{Main/Conjunto}
%\input{Main/Limpieza}
%\input{Main/Seleccionmodelo}
%\input{Main/Conclusiones}
%\input{Main/Evaluación}

\newpage
\printbibliography[heading=bibnumbered]

\end{document}
